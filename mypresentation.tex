Classifcation Metrics: Accuracy, Macro-Precision, Macro-Recall, Macro-averaged F1-score
Data splitting method: Hold-out

\begin{document} 

{Accuracy}
Metrics like accuracy are not well-suited for imbalanced datasets because
-> mainly influenced by the largest class 
-> correctly identifying minority classes is often more important in this context
-> not sufficient for evaluating model performance 
(Carvalho et al., 2025)
-> Useful as an overview, but it will not be used as a primary metric
Accuracy score:


{Macro-averaged F1-score, Macro-Recall and Macro-Precision}
Before calculating Macro F1-score, Macro-Precision and Macro-Recall need to be calculated
-> by averaging precision and recall across all classes 
-> each class has same importance 
-> minority classes are just as influential as majority classes
(Grandini et al., 2020)


By using the harmonic mean, 
-> the F1-score balances precision and recall 
-> both can contribute equally to the overall value
-> rewards models that achieve a good compromise between precision and recall rather than 
excelling in only one of them. 
(Grandini et al., 2020)
Macro-F1 score:
Macro-precison:
Macro-recall

{Hold-out splitting}
Dataset is divided into training data and  test data
-> test set is only used as final evaluation
-> efficient, classifier is trained only once
-> fairly simple concept
Hold-out splitting score:

{Literatur}
Carvalho, Miguel and Pinho, Armando and Bras, Susana.(2025). Resampling approaches to handle class 
imbalance: a review from a data perspective. Journal of Big Data. 12.10.1186/s40537-025-01119-4.
Grandini, Margherita and Bagli, Enrico and Visani, Giorgio. (2020). Metrics for Multi-Class 
Classification: an Overview. 10.48550/arXiv.2008.05756
\end{document}

