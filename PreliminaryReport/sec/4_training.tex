\section{Training and optimisation}
\label{sec:training}

\noindent\textit{(Research lead: Mehmet Berberoğlu)}

Training mirrors the staged pipeline described in the optimisation research. After the preprocessing stack produces normalized $64\times64$ faces, mini-batches travel through the modified ResNet-18 and the softmax outputs feed into weighted cross-entropy, while Adam keeps the gradient steps adaptive to the heterogeneous samples \cite{kingma2014adam}. Training runs for multiple epochs with checkpointing to enforce a dedicated validation split, thereby exposing overfitting tendencies that often arise from dataset-specific artefacts or the limited expressive content of fixed-resolution faces.

Class imbalance is mitigated through the loss weights: each rare emotion is assigned a greater contribution so that its misclassifications change the gradient significantly. If weighting proves insufficient to enforce parity, the plan includes more advanced sampling strategies (e.g., over-sampling minority classes or using focal loss variants). Adam remains the primary solver so that parameter updates adapt to the noisy, varied inputs without manual tuning \cite{fernando2021dwb}. Rectified Linear Unit (ReLU) activations throughout the network keep gradients well-behaved, while dropout in the final residual blocks and $L_2$ weight decay curb reliance on single neurons or runaway parameters when the training data are noisy or diverse \cite{nair2010rectified, srivastava2014dropout, loshchilov2017decoupled}. Together these measures prioritise stable, balanced learning rather than maximal but brittle accuracy.
