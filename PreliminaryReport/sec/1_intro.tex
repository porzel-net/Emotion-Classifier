\section{Introduction}
\label{sec:intro}

This report distills the collective research into a coherent strategy for the six-emotion classification task under the $64\times64$ constraint. The problem statement drives every design decision: the limited resolution reduces the signal available to represent subtle facial expressions, yet the downstream demo requires robust, interpretable outputs for happiness, surprise, sadness, anger, disgust, and fear. Each subsequent section unpacks one of the research threads that were developed in parallel—data sourcing and preprocessing, architecture design, training/optimisation, and evaluation—so that the next project milestones can inherit the same assumptions, trade-offs, and metrics.

The following pages therefore do not attempt to summarize implementation status, but rather present the technical rationale gathered from the research documents. This creates a shared vocabulary for the final paper and situates every architectural choice within the broader discussion of dataset heterogeneity, regularisation, and explainability. The conclusion then revisits the alternatives discussed in the research and suggests how the ongoing implementation work could leverage those insights.
